\documentclass[border=10pt]{standalone}
\usepackage{smartdiagram}
\usepackage[ngerman]{babel}
\usepackage[utf8]{inputenc} %UTF-Code für Umlaute
\usepackage[T1]{fontenc} %Trennung von Wörtern mit Umlauten
%\usepackage{lmodern}
%\usepackage{libertine}
\renewcommand*\familydefault{\sfdefault} %Serifenlose Schrift als Standard
%\usepackage{microtype}
\usepackage{verbatim}
\usesmartdiagramlibrary{additions}
\usetikzlibrary{fit}

\tikzstyle{container} = [draw, rectangle, semithick, %inner sep=0cm
]

%Aufzählungsstriche
\AtBeginDocument{
	\def\labelitemi{\normalfont\bfseries{--}}
}


\begin{document}
	
	\begin{tikzpicture}[
	every node/.style = {shape=circle, % is not necessary, default node's shape is rectangle
	%	rounded corners,
	%	draw= none	
		, semithick,
	%	text width=5cm,
		align=center,
		node distance=1.6cm
	}
	]
	\node (1)[draw
	,color=black!60!green
	]{1};
	\node (2)[right = of 1, draw]{2};
	\node (3)[right = of 2, draw
	,color=black!60!green
	]{3};
	
	\node (4)[below = of 1, draw]{4};
	\node (5)[right = of 4, draw
	, thick,color=blue
	]{5};
	\node (6)[right = of 5, draw]{6};
	
	\node (7)[below = of 4, draw]{7};
	\node (8)[right = of 7, draw
	,color=black!60!green
	]{8};
	\node (9)[right = of 8, draw]{9};
	

%Pfade

%Starke, direkte Beziehungen
	
	\path [->, thick
	, color=red
	] (1) edge (5);	
	\path [->, thick
	, color=red
	] (3) edge (5);
	\path [->, thick
	, color=red
	] (8) edge (5);
	
%\begin{comment}
% Die schwachen direkten Beziehungen
		\path [->, dashed, thick
	,color=red
	] (4) edge (5);
	\path [->, dashed, thick
	,color=red
	] (9) edge (5);
	

%\end{comment}	
	
%\begin{comment}
% Die starken indirekten Beziehungen	
	\path [->, thick] (1) edge (2);
	\path [->, thick] (2) edge (7);
	\path [->, thick] (7) edge (8);
	
	\path [->, thick] (8) edge (6);
	\path [->, thick] (6) edge (3);
	\path [->, thick, bend left=30] (6) edge (4);
	\path [->, thick] (9) edge (4);


%end{comment}
	
%\begin{comment}
% Die schwachen indirekten Beziehungen 	
	\path [->, dashed, thick] (4) edge (1);
	\path [->, dashed, thick] (1) edge (8);
	\path [->, dashed, thick,  bend left=30] (1) edge (3);
	\path [->, dashed, thick] (3) edge (4);
	\path [->, dashed, thick,  bend left=30] (7) edge (1);


%ä\end{comment}	
\end{tikzpicture}
	
\end{document}